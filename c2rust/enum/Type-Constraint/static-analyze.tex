\documentclass[peerreview, 10pt]{IEEEtran}

\usepackage{times}
\usepackage{graphicx}
\usepackage{proof}
\usepackage{amssymb}
\usepackage{amsthm}
\usepackage{amsmath}
\usepackage{amsfonts}
\usepackage{listings}
\usepackage{hyperref}
\usepackage{enumitem}
\usepackage{xcolor}
\usepackage{simplebnf}
\usepackage[margin=0.75in]{geometry}
\usepackage{framed}
\usepackage{stmaryrd}

\setlength{\jot}{6pt}

\definecolor{codegray}{rgb}{0.5,0.5,0.5}
\definecolor{codegreen}{rgb}{0,0.6,0}
\definecolor{codepurple}{rgb}{0.58,0,0.82}
\definecolor{backcolour}{rgb}{0.95,0.95,0.92}

\lstdefinelanguage{Rust}{
  keywords={break, const, continue, crate, else, enum, extern, false, fn, for, 
if, impl, in, let, loop, match, mod, move, mut, pub, ref, return, self, Self, 
static, struct, super, trait, true, type, unsafe, use, where, while, as, async, 
await, dyn},
  keywordstyle=\color{magenta},
  ndkeywords={u8, u16, u32, u64, usize, i8, i16, i32, i64, isize, f32, f64, str,
 String, char, bool, Option, Result, Some, None, Ok, Err, Box, Vec},
  ndkeywordstyle=\color{blue},
  identifierstyle=\color{black},
  sensitive=true,
  comment=[l]{//},
  morecomment=[s]{/*}{*/},
  commentstyle=\color{codegreen}\ttfamily,
  stringstyle=\color{codepurple}\ttfamily,
  morestring=[b]",
  morestring=[b]'
}

\lstdefinestyle{mystyle}{
    backgroundcolor=\color{backcolour},   
    commentstyle=\color{codegreen},
    keywordstyle=\color{magenta},
    numberstyle=\tiny\color{codegray},
    stringstyle=\color{codepurple},
    basicstyle=\ttfamily\footnotesize,
    breakatwhitespace=false,         
    breaklines=true,                 
    captionpos=b,                    
    keepspaces=true,                 
    numbers=left,                    
    numbersep=5pt,                  
    showspaces=false,                
    showstringspaces=false,
    showtabs=false,                  
    tabsize=2
}

\lstset{style=mystyle}

\newcommand{\C}[1]{{\small\tt #1}}
\newcommand{\Z}{\ensuremath{\mathbb{Z}}}
\newcommand{\lete}[3]{\ensuremath{\mathsf{let}\ #1\ \mathsf{=}\ #2\ \mathsf{in}\ #3}}
\newcommand{\lett}[4]{\ensuremath{\mathsf{let}\ #1\mathsf{:}\,#2\ \mathsf{=}\ #3\ \mathsf{in}\ #4}}
\newcommand{\binop}[2]{\ensuremath{#1\ \mathsf{\oplus}\ #2}}
\newcommand{\ternary}[3]{\ensuremath{\mathsf{if0}\ #1\ \mathsf{then}\ #2\ \mathsf{else}\ #3}}
\newcommand{\func}[2]{\ensuremath{\lambda #1.#2}}
\newcommand{\funct}[3]{\ensuremath{\lambda #1\mathsf{:}\,#2.#3}}
\newcommand{\apply}[2]{\ensuremath{#1\ #2}}
\newcommand{\raw}[1]{\ensuremath{#1\mathsf{.raw}}}
\newcommand{\arrowt}[2]{\ensuremath{{#1}\rightarrow{#2}}}
\newcommand{\checktype}[4]{\ensuremath{{check}(#1,\,#2)=(#3,\,#4)}}
\newcommand{\requiretype}[4]{\ensuremath{{require}(#1,\,#2,\,#3)=#4}}
\newcommand{\required}[1]{\ensuremath{#1\text{ is required}}}
\newcommand{\nrequired}[1]{\ensuremath{#1\text{ is not required}}}
\newcommand{\forcerequire}[2]{\ensuremath{#1\Rightarrow#2}}
\newcommand{\domain}[1]{\ensuremath{\text{Domain}(#1)}}
\newcommand{\cons}[1]{\ensuremath{\llbracket#1\rrbracket}}
\newcommand{\consp}[2]{\ensuremath{\langle#1,\,#2\rangle}}
\newcommand{\floor}[1]{\ensuremath{\lfloor#1\rfloor}}
\newcommand{\ceil}[1]{\ensuremath{\lceil#1\rceil}}
\newcommand{\enum}{\ensuremath{\epsilon}}
\newcommand{\ifrule}[2]{\ensuremath{#1&:&#2}}
\newcommand{\iffrule}[2]{\ensuremath{#1&:&#2}}

\begin{document}

\title{Enum Translation with Type Constraints}

\author{
    \IEEEauthorblockN{Minwook Lee}\\
    \IEEEauthorblockA{ryan-lee@kaist.ac.kr}
}

\maketitle

\section{Notation and Assumptions}

\begin{itemize}
  \item We consider a type as {\bf a set of values}. Therefore, ${\sf int}$ is equivalent to $\Z$.
  \begin{itemize}
     \item Every set operation can be applied to types as well. For example, ${\sf int} \cup \enum = \Z$ and $\enum\subseteq\Z$.
     \item Every unary or binary operation can be applied to types as well. For example: $$\tau_1\ {binop}\ \tau_2 = \{v_1\ {binop}\ v_2 \mid v_1 \in \tau_1 \land v_2 \in \tau_2\}.$$
  \end{itemize}
  \item Type constraint of the identifier $x$ is of the form $\cons{x} = \consp{\floor{x}}{\ceil{x}}$.
  \begin{itemize}
    \item $\floor{x}$ is a set of possible values that can be assigned to the identifier.
    \item $\ceil{x}$ is the required type for $x$ based on the type annotation and its usage context.
    \item Every set operation can be applied to type constraints as well. For example:
    \begin{itemize}
      \item $\cons{x}\subseteq\cons{y} \text{ iff } \floor{x} \subseteq \floor{y} \land \ceil{x} \subseteq \ceil{y}$
      \item $\cons{x} \cup \cons{y} = \consp{\floor{x} \cup \floor{y}}{\ceil{x} \cup \ceil{y}}$
      \item $\cons{x}\ binop\ \cons{y} = \consp{\floor{x}\ binop\ \floor{y}}{\ceil{x}\ binop\ \ceil{y}}$
    \end{itemize}
  \end{itemize}
\end{itemize}

\section{Basic MIR}

\subsection{Abstract Syntax}

\begin{figure}[ht]
\centering
\begin{framed}
\[
  \begin{array}{@{}rcl@{}}
    n &\in& \mathbb{Z} \\
    x,y &\in& \mathit{Id} \\
    f &:=& \overline{B:b};\overline{x:\tau} \\
    b &:=& \overline{s}; t \\
    s &:=& x=n\ |\ x=y\ |\ x={\it unop}\ y\ |\ x=y_1\ {\it binop}\ y_2 \\
    t &:=& {\sf goto}\ B\ |\ {\sf switch}\ x\ \overline{n:B}\ |\ {\sf return} \\
    \epsilon &\in& \mathit{Enum} \\
    \tau &:=& {\sf int}\ |\ \epsilon \\
  \end{array}
\]
\end{framed}
\caption{Basic MIR Abstract Syntax}
  \label{fig:basic-absyn}
\end{figure}

The abstract syntax of the basic MIR is shown in Fig~\ref{fig:basic-absyn}.
%
In MIR, a function $f$ is represented with the type
\C{Body}\footnote{\url{https://doc.rust-lang.org/beta/nightly-rustc/rustc\_middle/mir/struct.Body.html}},
which denotes a control flow graph (CFG) consisting of multiple basic blocks.
%
Each basic block $b$ is represented with the type
\C{BasicBlockData}\footnote{\url{https://doc.rust-lang.org/beta/nightly-rustc/rustc\_middle/mir/struct.BasicBlockData.html}}
and consists of multiple statements followed by a single terminator.
%
In a basic block, statements do not have any control flow effect, and only
terminators affect the control flow.
%
A basic block is associated with a unique identifier $B$, represented with the
type
\C{BasicBlock}\footnote{\url{https://doc.rust-lang.org/beta/nightly-rustc/rustc\_middle/mir/struct.BasicBlock.html}}.
%
A statement $s$ has the type
\C{Statement}\footnote{\url{https://doc.rust-lang.org/beta/nightly-rustc/rustc_middle/mir/struct.Statement.html}}.
%
While various kinds of statements exist in MIR, we only need to consider
assignments.
%
The left-hand side of an assignment is a variable $x$, and the right-hand side
is either an integer $n$, a variable, a unary operation, or a binary operation.
%
A terminator $t$ has the type
\C{Terminator}\footnote{\url{https://doc.rust-lang.org/beta/nightly-rustc/rustc\_middle/mir/struct.Terminator.html}}
and is either a jump to another basic block, a switch, or a return.
%
When returning, the return value is always the value stored in the variable
named \C{\_0}.
%
If a function has $n$ parameters, their values are stored in the variables
\C{\_1}, \C{\_2}, \ldots, \C{\_n}.
%
Note that a function declares the type of each variable as well.
%
A type $\tau$ is either \C{int} or a C \C{enum} type $\epsilon$.

\subsection{Generating Type Constraints}

The rules for type constraints in basic MIR are shown in Fig~\ref{fig:basic-type-constraints}.

\begin{figure}[ht]
\begin{framed}
For the function $f$:
\[
\begin{array}{@{}ccl@{}}
  \ifrule{\overline{B:b};\overline{x:\tau}}{\ceil{\overline{x}}\subseteq\overline{\tau}}\\
\end{array}
\]
For the statement $s$:
\[
\begin{array}{@{}ccl@{}}
  \ifrule{x=n}{n\in\floor{x}}\\
  \ifrule{x=y}{\cons{y}\subseteq\cons{x}}\\
  \ifrule{x={\it unop}\ y}{unop\ \floor{y}\subseteq\floor{x}}\\
  \ifrule{x=y_1\ {\it binop}\ y_2}{\floor{y_1}\ binop\ \floor{y_2}\subseteq\floor{x}}
\end{array}
\]
\end{framed}
\caption{Basic MIR Type Constraints}
\label{fig:basic-type-constraints}
\end{figure}

\subsection{Resolving Type Constraints}

We resolve the type constraints with the following principles:

\begin{itemize}
  \item We always choose the smallest set for $\floor{x}$ that satisfies the type constraints.
  \begin{itemize}
    \item This increases the coverage where the enum translation can be applied.
  \end{itemize}
  \item We always choose the largest set for $\ceil{x}$ that satisfies the type constraints.
  \begin{itemize}
    \item This prevents the redundant enum translation.
  \end{itemize}
\end{itemize}

\subsection{Examples}
Assume $\enum = \{0,1\}$.
\subsubsection{Simple case}
The following is an example of a basic MIR function:
\[
  x:\enum;\ y:\Z;\ \{y=1;\ x=y;\ \sf return\}
\]

We can derive the following conditions:

\[
\begin{array}{@{}ccl@{}}
  \ifrule{x:\enum}{\ceil{x}\subseteq\enum}\\
  \ifrule{y:\Z}{\ceil{y}\subseteq\Z}\\
  \ifrule{y=1}{1\in\floor{y}}\\
  \ifrule{x=y}{\cons{y}\subseteq\cons{x}}
\end{array}
\]

By resolving the type constraints we have:
\[
\cons{x}=\consp{\{1\}}{\enum}\quad\cons{y}=\consp{\{1\}}{\enum}.
\]

Now we deduce that $\floor{x}\subseteq\ceil{x}$ and $\floor{y}\subseteq\ceil{y}$. Therefore, we now safely modify the function to:
\[
\cdots;\ y:\enum;\ \cdots.
\]

Also, the translation doesn't introduce the unsafe code since all required types are valid Rust enum types.

\subsubsection{Solving more cases with type constraints}
The following is an example of a basic MIR function showing how deriving value constraints would work in case where subtype inference could not be applied:
\[
  x:\enum;\ y:\Z;\ \{y=-1;\ x=-y;\ \sf return\}
\]

We can derive the following conditions:

\[
\begin{array}{@{}ccl@{}}
  \ifrule{x:\enum}{\ceil{x}\subseteq\enum}\\
  \ifrule{y:\Z}{\ceil{y}\subseteq\Z}\\
  \ifrule{y=-1}{-1\in\floor{y}}\\
  \ifrule{x=-y}{-\floor{y}\subseteq\floor{x}}
\end{array}
\]

By resolving the type constraints we have:
\[
\cons{x}=\consp{\{1\}}{\enum}\quad\cons{y}=\consp{\{-1\}}{\Z}.
\]

Now we deduce that all $\floor{x}\subseteq\ceil{x}$ and $\floor{y}\subseteq\ceil{y}$. Therefore, we can translate the integer type to the Rust enum type. However, this procedure would introduce unsafe code on $x=-y$ while translating the type of $x$ to be $\enum$.

\section{Adding pointers}

\subsection{Abstract Syntax}

The abstract syntax of MIR with pointers is shown in Fig~\ref{fig:mir-pointers-absyn}.

\begin{figure}[ht]
\centering
\begin{framed}
\[
  \begin{array}{@{}rcl@{}}
    s &:=& \cdots\ |\ x=\&y\ |\ x=*y\ |\ *x=y \\
    \tau &:=& \cdots\ |\ *\tau \\
  \end{array}
\]
\end{framed}
\caption{MIR Abstract Syntax with Pointers}
\label{fig:mir-pointers-absyn}
\end{figure}

We denote the abstract cell $c\in\mathit{Cell}$ as the set of all locations while $\cons{c}=\consp{\floor{c}}{\ceil{c}}$ denotes the type constraint of the value which is located on $c$. The pointer type $*\tau$ denotes the set of cells that can be dereferenced to a value of type $\tau$. If $x$ can be dereferenced to a value of type $\tau_1$ and $\tau_1\subseteq\tau_2$, then $x$ can also be dereferenced to a value of type $\tau_2$. This means that $*\tau_1\subseteq *\tau_2$.

\subsection{Generating Type Constraints}

We choose an Andersen-style pointer analysis, since we do not expect real-world C code to use extensive pointer operations that would require flow-sensitive analysis for enum types.

The rules for type constraints in MIR with pointers are shown in Fig~\ref{fig:mir-pointers-type-constraints}.

\begin{figure}[ht]
\begin{framed}
For the statement $s$:
\[\cdots\]
\[
\begin{array}{@{}lclccc@{}}
  \ifrule{x=\&y}{\forall\tau,&y\subseteq \floor{x}&\land&\tau\subseteq\ceil{y}\Rightarrow*\tau\subseteq\ceil{x}}\\
  \ifrule{x=*y}{\forall\tau\forall c,&
    c\in \floor{y}\Rightarrow\cons{c}\subseteq\cons{x}&\land&
    *\tau\subseteq\ceil{y}\Rightarrow\tau\subseteq\ceil{x}
  }\\
  \ifrule{*x=y}{\forall\tau\forall c,&
    c\in \floor{x}\Rightarrow\cons{y}\subseteq\cons{c}&\land&
    \tau\subseteq\ceil{y}\Rightarrow*\tau\subseteq\ceil{x}
  }
\end{array}
\]
\end{framed}
\caption{MIR Type Constraints with Pointers}
\label{fig:mir-pointers-type-constraints}
\end{figure}

\subsection{Examples}

Assume $\enum = \{0,1\}$.
The following is an example of a basic MIR function with pointers:
\[
  x:*\Z;\ y:\Z;\ z:\enum;\ \{y=1;\ x=\&y;\ z=*x;\ \sf return\}
\]

Each statement and variable declaration generates the following type constraints:
\[
\begin{array}{@{}ccl@{}}
  \ifrule{x:*\Z}{\ceil{x}\subseteq *\Z}\\
  \ifrule{y:\Z}{\ceil{y}\subseteq \Z}\\
  \ifrule{z:\enum}{\ceil{z}\subseteq \enum}\\
  \ifrule{y=1}{1\in\floor{y}}\\
  \ifrule{x=\&y}{\forall\tau,\ y\subseteq \floor{x}\ \land\ \tau\subseteq\ceil{y}\Rightarrow*\tau\subseteq\ceil{x}}\\
  \ifrule{z=*x}{\forall\tau\forall c,\ c\in \floor{x}\Rightarrow\cons{c}\subseteq\cons{z}\ \land\ *\tau\subseteq\ceil{x}\Rightarrow\tau\subseteq\ceil{z}}
\end{array}
\]

Now we resolve the type constraints as follows:

\[
\begin{aligned}
  &\floor{y}=\{1\}\\
  &\floor{x}=\{y\}\\
  &\cons{y}\subseteq\cons{z}\\
  &\tau\subseteq\ceil{y}\Rightarrow *\tau\subseteq\ceil{x}\Rightarrow\tau\subseteq\ceil{z}\subseteq\enum\\
  \therefore\ &\ceil{x}\subseteq*\enum\ \land\ \ceil{y}\subseteq\enum\ \land\ \ceil{z}\subseteq\enum
\end{aligned}
\]

By resolving the type constraints we have:

\[
\cons{x}=\consp{\{y\}}{*\enum}\quad\cons{y}=\consp{\{1\}}{\enum}\quad\cons{z}=\consp{\{1\}}{\enum}.
\]

\section{Adding structs and arrays}

\subsection{Abstract Syntax}
The abstract syntax of MIR with structs and arrays is shown in Fig~\ref{fig:mir-structs-arrays-absyn}.
\begin{figure}[ht]
\centering
\begin{framed}
\[
  \begin{array}{@{}rcl@{}}
    F &\in& \mathit{Field} \\
    S &\in& \mathit{Struct} \\
    s &:=& \cdots\ |\ x.F=y\ |\ x=y.F\ |\ x[x']=y\ |\ x=y[y'] \\
    \tau &:=& \cdots\ |\ S\ |\ \tau[] \\
    p &:=& \overline{S \{ \overline{F:\tau} \}}; \overline{f} \\
  \end{array}
\]
\end{framed}
\caption{MIR Abstract Syntax with Structs and Arrays}
\label{fig:mir-structs-arrays-absyn}
\end{figure}



\end{document}